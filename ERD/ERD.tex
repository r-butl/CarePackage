\documentclass{article}
\usepackage{graphicx} % Required for inserting images
\usepackage{amsmath}
\usepackage[margin=1in]{geometry}
\usepackage{listings}

% Title setup
\title{E.R.D.}
\author{Lucas Butler}
\date{\today}

%Document
\begin{document}

\maketitle

\section*{Glossary}
\label{sec:glossary}

\section{Product Definition}

\subsection{Summary}
\quad In the United States, Cardiac disease is produces a significant number of casualties, nearly 700,000 deaths a year[1]. Currently, anomalies such as irregular heart rhythms  (arrhytmia) and blood flow blockages (myocardial infarction) are monitored through the collection and classification of Electrocardiogram signals. This signal is typically collected in a clinical setting, through electrodes and wires, under the supervision of Doctors and Nurses. Recent technological advancements 

\subsection{Identification}

\quad
\subsection{Users}

This project contributes to society by improving the efficiency of triage within the health care system, in addition to making cardiac telemetry more accessible. A versatile ECG signal classification system will serve a wide variety of health professionals, patients at-risk, and everyday health enthusiasts.

This technology will be heavily utilized by hospitals and clinics. Centralized monitoring will enable a more informed view for medical professionals to utilize in their decision making. Automating the collection and detection of anomalies will improve prioritization of patients and provide more opportunity to prevent failure, significantly adding to the health-care effort. Practitioners will benefit from additional medical data when researching root causes and fine tuning their assessment policies. Introducing this streamlined approach will empower medical professionals to excel in their responsibilities.

This technology will significantly improve the patient experience. Those at-risk are typically worried about abnormalities and feel tied to hospitals or clinics for consistent updates. Unrestricted cardiovascular monitoring will relieve this stress by facilitating patient autonomy and fostering a deeper engagement with more enjoyable daily tasks. With improved insight, those at-risk will be able to assess the challenges they face during various activities, introducing a drastic improvement in the average quality of life. 

Everyone, including those who are not considered at-risk or health professionals, will benefit. Providing deeper insights into cardiovascular health will allow individuals to make better decisions about their eating, exercising and sleeping habits. Athletes and fitness enthusiasts will utilize this information to prep for important matches and improve training regiments. Post-op surgery patients will benefit from smoother recovery, and the Eldery will rest assured that they are in good health.

\subsection{Interfaces}
\subsection{User Requirements}
\subsection{Customer Needs}

\section{Project Definition}

\subsection{Summary}
The focus of the project will begin with the sampling of the ECG signal, an implementation of a Median filter algorithm to remove noise, a Discrete Wavelet Transform algorithm to isolate the target frequency band, AES-256 encryption

\subsection{Constraints and Limitations}
The implementation of this system will be limited due to time constraints. The implementation of a product this size and scale will take many considerations and revises. Knowing this, it is important to first identify time as the main restriction. 
\subsection{Assumptions and Dependencies}
\subsection{Architecture}
\subsection{Design Alternatives}
\subsection{Specifications}
\subsection{Project Schedule}
\subsection{Prototype Costs}

\section{References and Standards}
[1] Centers for Disease Control and Prevention. "Heart Disease Facts." Available: https://www.cdc.gov/heartdisease/facts.htm
[2] American College of Cardiology. "Chronic Coronary Disease Guidelines (2023)." Available: https://www.acc.org/Guidelines/Hubs/Chronic-Coronary-Disease
\section{Appendix}
\end{document}